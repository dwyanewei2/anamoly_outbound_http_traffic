\begin{abstract}

We present a counterfeit fingerprint detection to identify anomalous outbound HTTP communication, which focuses on fingerprints for each browser running on a client-side host. In recent years, malware and backdoor usually hide their malicious activities in communications with HTTP protocol. However, the hidden technique hacker used is counterfeit fingerprint which can easily evade current detections. We evaluate our approach with real-world data from international institute, and the experimental results show that our method achieves an accuracy rate of 0.99\%, especially counterfeit fingerprint attacks could be all detected by our approach. Moreover, we also compare our solution with DECANTeR \cite{bortolameotti2017decanter}, which detects outbound HTTP communications by focusing on benign HTTP traffic with fingerprint technique. The results show that our approach outperforms DECANTeR in terms of accuracy rate and false positive rate, and this paper shows that our approach is potential to detect data leakage using counterfeit technique.

\end{abstract}

\begin{IEEEkeywords}
Anomaly Detection, Data Exfiltration, Data Leakage, Application Fingerprinting, Network Security
\end{IEEEkeywords}
