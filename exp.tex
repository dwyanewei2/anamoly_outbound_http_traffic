\section{Experiment Results}

{\color{red}{Tony, pliz write overview here!!!}} \\
In this section, we would describe the datasets that we used to perform our experiments. For performing our experiments we have used two different datasets, simulated data and real-world data. We use simulated data to evaluate the detection performance of our system and compare with DECANTeR. \\


\subsection{Experimental Settings}

{\color{red}{Tony, pliz write like the followings in this subsection!}}\\

\subsubsection{simulated data} 
We build a botnet malware and monitor its HTTP outbound. As we know, early botnets generally used Internet Relay Chat (IRC) channel to communicate C\&C server. In recent years, botnets also start to communicate C\&C server through HTTP protocol. Moreover, we need to build a botnet with the spoofing headers which can evade other detection systems. That is why we collect three different simulated botnets traffic. For dataset\_01 is the botnet without spoofing headers. We collect the Infected host's HTTP outbound traffic .The Dataset\_02 consists of the botnet traffic with simple spoofing. It means that our botnet sends the requests to web-like user agent through HTTP protocol. And the dataset\_03 is totally modifying the headers information. The malware can detect which browser is used by the user, and fills the User-Agent with that specific browser. Moreover, it also fills up some common requests header fields, like Accept, Accept-Encoding, Accept-Language, Referer. The datasets information is in table~\ref{tbl:db_01}.

\subsubsection{real-world data}

We have collected the traffic from more than hundreds of machines in the technology industry. Users of these machines include accountants, engineers, sales executive, and administrative personnel. Since the users vary from different occupation, the data has complexity. We have collected data by \textit{tcpdump} with three working days. Besides, we have set the \textit{tcpdump} that only collected the traffic with the outbound HTTP protocol.

The real world dataset has splited to two set, one is training and the other is testing. Training set has covered first few days and testing set is the last of traffic.

Malware analyzed in following experiments are malicious behavioral sequences from collected malicious executable files in real world.  In proposed method implementation, collected malware were paresed and information of four major fields in table~\ref{tbl:db_01}. Table~\ref{tbl:db_01} also lists the the number of malware families, the size of gathered data, and operation types in malicious behavioral sequences. 


\subsection{Evaluation Metrics}

{\color{red}{Tony, pliz replace malware to fingerprint!!!}}

Essentially, malware clustering is a multiple classification problem aim to identify malware comes from which families. Four well-known metrics for evaluating effectiveness of proposed method are adopted as followings: ``true positive''($TP$) means the number of malware which belong to same malware families. ``False negative''($FN$) is the number of malware which's families are wrongly predicted. Similarly, ``true negative''($TN$) means the number of malware which aren't same families and being viewed as others, while ``false positive''($FP$) is the number of false alarms that other families' malware being detected as the same ones. Based on accumulation of $TP, FN, TN$, and $FP$, one extended metrics ($accuracy$) popularly used in machine learning problems are also adopted here to evaluate proposed method and listed in equations below. Note that the optimal $accuracy$ of $1.0$ means all of malware are successfully classified by proposed approach. 

\begin{eqnarray}
\label{eq:accuracy}
Accuracy = \frac{TP+TN}{TP+TN+FP+FN}
\end{eqnarray}

\subsection{Effectiveness Analysis}

{\color{red}{Tony, pliz show your exp graph and table here!!!}}\\

\begin{table}[]
\centering
\caption{Simulated Data}
\label{tbl:db_01}
\begin{tabular}{|c|c|c|c|}
\hline\hline
Dataset                    & Dataset    & HTTP Requests        & Performance \\ \hline
\multirow{2}{*}{Dataset\_01}  & Decanter   & \multirow{2}{*}{220} & 100\%       \\ \cline{2-2} \cline{4-4} 
                           & Our System &                      & 100\%       \\ \hline
\multirow{2}{*}{Dataset\_02} & Decanter   & \multirow{2}{*}{216} & 100\%       \\ \cline{2-2} \cline{4-4} 
                           & Our System &                      & 100\%       \\ \hline
\multirow{2}{*}{Dataset\_03} & Decanter   & \multirow{2}{*}{216} & 0\%         \\ \cline{2-2} \cline{4-4} 
                           & Our System &                      & 100\%       \\ \hline\hline
\end{tabular}
\end{table}

\subsection{Limitation and Future Work}
