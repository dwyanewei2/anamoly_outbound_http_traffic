\section{Experiment Results}

In this section, we would describe the datasets that we used to perform our experiments. For performing our experiments we have used two different datasets, simulated data and real-world data. We use simulated data to evaluate the detection performance of our system and compare with DECANTeR. \\


\subsection{Experimental Settings}

In the following, we describe the detail about datasets we used to evaluate our system.\\

\begin{table}[!h]
\centering
\caption{Overview of the Datasets}
\label{tbl:db_01}
\begin{tabular}{|c|c|c|c|c|}
\hline
Dataset         & Features & Type            & All Samples & Malicious Samples \\ \hline
Industy\_flow01 & Packets  & Malicious Flows & 1690869     & N/A               \\ \hline
Industy\_flow02 & Packets  & Malicious Flows & 68234       & N/A               \\ \hline
Dataset\_01     & Packets  & Botnet          & 220         & 220               \\ \hline
Dataset\_02     & Packets  & Botnet          & 216         & 216               \\ \hline
Dataset\_03     & Packets  & Botnet          & 1045         & 168               \\ \hline
\end{tabular}
\end{table}

\subsubsection{simulated data} 
We build a botnet malware and monitor its HTTP outbound. As we know, early botnets generally used Internet Relay Chat (IRC) channel to communicate C\&C server. In recent years, botnets also start to communicate C\&C server through HTTP protocol. Moreover, we need to build a botnet with the spoofing headers which can evade other detection systems. That is why we collect three different simulated botnets traffic. For dataset\_01 is the botnet without spoofing headers. We collect the Infected host's HTTP outbound traffic .The Dataset\_02 consists of the botnet traffic with simple spoofing. It means that our botnet sends the requests to web-like user agent through HTTP protocol. And the dataset\_03 is totally modifying the headers information. The malware can detect which browser is used by the user, and fills the User-Agent with that specific browser. Moreover, it also fills up some common requests header fields, like Accept, Accept-Encoding, Accept-Language, Referer. The datasets information is in table~\ref{tbl:db_01}.\\

\subsubsection{real-world data}

We have collected the traffic from more than hundreds of machines in the technology industry. Users of these machines include accountants, engineers, sales executive, and administrative personnel. Since the users vary from different occupation, the data has complexity. We have collected data by \textit{tcpdump} with three working days. Besides, we have set the \textit{tcpdump} that only collected the traffic with the outbound HTTP protocol.

The real world dataset has split into two sets, one is training and the other is testing. Training set has covered first few days and testing set is the last of traffic. For training set contains 1690869 HTTP requests, it spent about two working days. And the testing set contains 68234 HTTP requests with one working day.

In the following experiments, we would use the dataset that describes in the above section. First, we show the result of header spoofing cases and compare with the system, which calls DECANTeR. Then we use our system to detect the malicious flow from the real world dataset.


\subsection{Evaluation Metrics}

Essentially, Our system is a flow filter aim to identify the suspicious requests with the headers field. Four well-known metrics for evaluating the effectiveness of proposed method are adopted as followings: ``true positive''($TP$) means the number of normal requests which belong to normal traffic. ``False negative''($FN$) is the number of normal traffic which's results are wrongly predicted. Similarly, ``true negative''($TN$) means the number of abnormal traffic and the system predict it as malicious requests, while ``false positive''($FP$) is the number of abnormal traffic that the system predicts it as normal traffic. Based on the accumulation of $TP, FN, TN$, and $FP$, one extended metrics ($accuracy$) popularly used in machine learning problems are also adopted here to evaluate proposed method and listed in equations below. Note that the optimal $accuracy$ of $1.0$ means all of the malware are successfully picked out by the proposed approach. 


\begin{eqnarray}
\label{eq:accuracy}
Accuracy = \frac{TP+TN}{TP+TN+FP+FN}
\end{eqnarray}

\subsection{Effectiveness Analysis}

Just as we know, malware can easily modify the HTTP headers. Hence, we build three similar malware to evaluate our approach. With these botnets, they all have the same purpose. They would steal some sensitive information (such as OS information, system account, and the password) and send requests to the C\&C server periodically waiting for commands to execute. The difference between them is the degree of the spoofing HTTP headers. The botnet in dataset\_01 does not spoof any HTTP headers. We fill up with the empty to the User-Agent field. The botnet in datset\_02 only simply sets the User-agent as a common browser which calls IE. In the dataset\_03's botnet would specifically detect the victim's browser version, system language and then fill them into the HTTP header fields. In addition, for the reference field in HTTP headers we default to point to the google website. The result show in table~\ref{tbl:db_02}, we can see both detection system has the great performance with dataset\_01 and dataset\_02. However, we can notice that our system has a better performance for botnets that based on advanced spoofing methods.
\\

\begin{table}[!h]
\centering
\caption{Simulated Data}
\label{tbl:db_02l}
\begin{tabular}{|c|c|c|c|c|c|c|c|}
\hline
\multirow{2}{*}{Dataset}     & \multirow{2}{*}{System} & \multirow{2}{*}{HTTP Requests} & \multicolumn{4}{c|}{Evaluation Metrics} & \multirow{2}{*}{Accuracy} \\ \cline{4-7}
                             &                         &                                & TP       & TN       & FP       & FN     &                           \\ \hline
\multirow{2}{*}{Dataset\_01} & Decanter                & \multirow{2}{*}{220}           &          &          &          &        &                           \\ \cline{2-2} \cline{4-8} 
                             & Our System              &                                &          &          &          &        &                           \\ \hline
\multirow{2}{*}{Dataset\_02} & Decanter                & \multirow{2}{*}{216}           &          &          &          &        &                           \\ \cline{2-2} \cline{4-8} 
                             & Our System              &                                &          &          &          &        &                           \\ \hline
\multirow{2}{*}{Dataset\_03} & Decanter                & \multirow{2}{*}{1045}          & 869      & 0        & 168      & 8      & 0.8315                    \\ \cline{2-2} \cline{4-8} 
                             & Our System              &                                & 869      & 168      & 0        & 8      & 0.9923                    \\ \hline
\end{tabular}
\end{table}


\subsection{Limitation and Future Work}
