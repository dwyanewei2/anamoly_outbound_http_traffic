\section{Conclusion}

In this work we have shown how HTTP-based applications can be fingerprinted and used to detect anomalous communications. This technique can be used without using malicious data during the training phase, therefore it avoids any possible bias from specific malware samples. Moreover, the proposed technique detects anomalous communication independently from their payload, thereby being a promising solution for data exfiltration and unknown malware. This distinguishes our work from most of the existing solutions, which often model network traffic to detect specific attacks or malware behavior (extracted from clusters of known malware), or tries to identify sensitive data within the network payload. We have implemented this technique in a system called DECANTeR, and we have evaluated it, showing better detection performance than other state of the art solutions.
